%Disciplina Projetos de Automação
%Turma: 
%Integrantes do grupo:
%Ano de 2016

\documentclass[10pt]{article}  
%% Pacotes utilizados para a produção do relatório
\usepackage[portuguese]{babel} % Indica que escrevemos português
\usepackage[utf8]{inputenc} % Indica que a codificação é ISO-8859-1(latin1)  o utf8  
\usepackage{amsmath} % Comandos extras para matemática
\usepackage{amssymb} % Símbolos matemáticos
\usepackage{graphicx} % Incluir imagens no LaTeX
\usepackage{color} % Para colorir texto
\usepackage{subfigure} % subfiguras
\usepackage{float} % Podemos usar o especificador [H] nas figuras para que fiquem na posição desejada
\usepackage{capt-of} % Permite usar etiquetas fora de elementos flutuantes
% (etiquetas de figuras)
\usepackage{sidecap} % Para colocar o texto das imagens ao lado delas
	\sidecaptionvpos{figure}{c} % Para que o texto de alinhe ao centro e verticalmente
\usepackage{caption} % Para poder colocar a numeração nas figuras
\usepackage{anysize}% Para personalizar os tamanhos das margens
	\marginsize{2cm}{2cm}{2cm}{2cm} % Esquerda, direita, acima, abaixo
\usepackage{appendix}
	\renewcommand{\appendixname}{Apêndices}
	\renewcommand{\appendixtocname}{Apêndices}
	\renewcommand{\appendixpagename}{Apêndices} 
\usepackage[colorlinks=true,plainpages=true,citecolor=blue,linkcolor=blue]{hyperref} % Para que os hypertextos apareçam coloridos.
\usepackage{fancyhdr} 
	\pagestyle{fancy}
	\fancyhf{}
	\fancyhead[L]{\footnotesize IFMG-Ouro Preto} % Cabeçalho à esquerda
	\fancyhead[R]{\footnotesize CODAAUT}   % Cabeçalho à direita
	\fancyfoot[R]{\footnotesize Relatório Final}  % Rodapé à direita
	\fancyfoot[C]{\thepage}  % Rodapé centro
	\fancyfoot[L]{\footnotesize Automação Industrial} % Rodapé à esquerda
	\renewcommand{\footrulewidth}{0.4pt}
\usepackage{listings} % Para usar código fonte
	\definecolor{dkgreen}{rgb}{0,0.6,0} % Definimos cores para usar no código
	\definecolor{gray}{rgb}{0.5,0.5,0.5} 
% configuração para a linguagem de programação que desejamos utilizar
	\lstset{language=Matlab,
   	keywords={break,case,catch,continue,else,elseif,end,for,function,
    	  global,if,otherwise,persistent,return,switch,try,while},
   	basicstyle=\ttfamily,
   	keywordstyle=\color{blue},
   	commentstyle=\color{red},
   	stringstyle=\color{dkgreen},
   	numbers=left,
   	numberstyle=\tiny\color{gray},
   	stepnumber=1,
   	numbersep=10pt,
   	backgroundcolor=\color{white},
   	tabsize=4,
   	showspaces=false,
   	showstringspaces=false}
%%%%%%%%%%%%%%%%%%%%%%%%%%%%%%%%%%%%%%%%%%%%%%%%%%%%%%%%%%%%%%%%%%%%%%%%%%%%%%%%%%%%%%%%%%%%%%%%%%%%%%%%%%%%%%%%%%%%%%%%%%%%%%%%%%%%%%%%%%%%%%%%%%%%%%%%%%%%%%%%%%%%%%%%%%%%%%%%%%%%%%%%%%%%%%%%%%%%%%%%%%%%%%%%%%%	

\title{Relatório de Projetos de Automação}

%%%%%%%%%%%%%%%%%%%%%%%%%%%%%%%%%%%%%%%%%%%%%%%%%%%%%%%%%%%%%%%%%%%%%%%%%%%%%%%%%%%%%%%%%%%%%%%%%%%%%%%%%%%%%%%%%%%%%%%%%%%%%%%%%%%%%%%%%%%%%%%%%%%%%%%%%%%%%%%%%%%%%%%%%%%%%%%%%%%%%%%%%%%%%%%%%%%%%%%%%%%%%%%%%%%	

\begin{document}
\begin{center}														
\newcommand{\HRule}{\rule{\linewidth}{0.5mm}} % \left

\begin{minipage}{0.48\textwidth} \begin{center}
	%\includegraphics[scale = 0.63]{Imagens/Logotipo_Campus_OP_sem_fundo}
\end{center}
\end{minipage}

\vspace*{1.5cm} % espaço dado no texto na vertical
\textsc{\huge Non de\\ \vspace{5px}  Automação Industrial}\\[1.5cm]	
\textsc{\LARGE Disciplina: Projetos de Automação }\\[1.5cm]									
\vspace*{1.5cm}																	
\HRule \\[0.5cm]
{ \Huge \bfseries Título - Nome principal do projeto }\\[0.2cm]
{ \huge \bfseries Subtítulo} 													
\HRule \\[1.5cm]																

\begin{minipage}{0.46\textwidth}
	\begin{flushleft} \large
		\emph{Autores:}\\	
		Autor 1\\
		Autor 2\\
		Autor 3\\ 
		Autor 4
	\end{flushleft}																
\end{minipage}		
%%%
\begin{minipage}{0.52\textwidth}		
\vspace{-0.6cm}
	\begin{flushright} \large															\emph{Funções:} \\														
		Função 1 \\
		Função 2 \\
		Função 3 \\
		Função 4 
	\end{flushright}															
\end{minipage}	
\vspace*{1cm}
\vspace{2cm} 																				
\begin{center}																		{\large \today}
\end{center}												  					
\end{center}							 																									
\newpage
\textbf{\Huge Resumo} 
\\
\linebreak
\textbf{Palavras-chave:} escolha três palavras-chave para o seu projeto.
O resumo deve conter até 400 palavras, descrevendo de forma breve todo o projeto implementado por vocês.

\newpage
\textbf{\Huge Abstract} 
\\
\linebreak
\textbf{Keywords:} escolha três palavras-chave para o seu projeto.

Aqui deve ser colocado um resumo em inglês.

Faça primeiro o resumo na página seguinte, em português e depois volte aqui e coloque o abstract em inglês! Vamos treinar nosso inglês praticando!

\newpage																
\tableofcontents  

\newpage
\section{Introdução}

Deve conter uma breve descrição do projeto.
Este texto deve ser capaz de contextualizar o leitor com relação ao seu projeto.

\subsection{Definição do problema a ser resolvido}

A automação de um processo deve servir a um propósito, que é resolver e/ou otimizar um problema/processo. Descreva aqui qual problema/processo seu projeto resolve/otimiza.

\subsection{Revisão bibliográfica}

O referencial teórico é a base que sustentará o projeto desenvolvido pelo grupo. 

É necessário conhecer o que já foi desenvolvido por outros projetistas. Assim, o estudo da literatura contribui para o projeto em muitos sentidos: na definição dos objetivos do trabalho, nas construções teóricas, no planejamento, nas comparações e na validação dos projetos.

De forma simplificada, seguem as orientações para a escrita da revisão bibliográfica:

\begin{enumerate}
\item Defina o tema do seu projeto.
\item Reúna a bibliografia. Reúna pelo menos 03 (três) trabalhos para ter uma visão panorâmica do assunto.
\item Analise as referências dos trabalhos selecionados anteriormente e identifique a estrutura hierárquica do assunto de pesquisa. A estrutura hierárquica vai do assunto mais geral ao mais específico. Sugestões comuns para todos os grupos, que devem ser tratadas sob a ótica do projeto de cada grupo:
\subitem{Arduíno}
\subitem{Aplicativos}
\subitem{Controle remoto de dispositivos via web}
\subitem{Dispositivos de entrada (sensores) e saída (acionamentos)}
\subitem{Interfaces de comunicação entre os dispositivos}

\item{Leia a bibliografia reunida com atenção e liste as idéias principais.}
\item{Identifique as idéias principais a serem aproveitadas em seu trabalho. Não se esqueça de indicar as fontes de cada idéia.}
\item{Conclua o referencial teórico identificando as principais idéias discutidas no seu texto e apontando para as questões de pesquisa em aberto na literatura.}
\end{enumerate}

\subsection{Objetivo}

Um parágrafo sucinto e claro esclarecendo o objetivo geral do projeto.

\subsubsection{Objetivos específicos}

Incluir os objetivos específicos para que o objetivo geral seja alcançado. Aqui, cabem as frases escritas com os seguintes verbos no infinitivo: realizar, adaptar, selecionar, construir, implementar, validar, estudar, utilizar, dentre outros.

\subsection{Motivação}

Neste item deve ser descrita a justificativa do projeto que o grupo propõe. Qual a real motivação para que este projeto seja implementado? Social, ambiental, econômica, cultural? 

\section{Metodologia} 

Como a implementação do projeto é abordada pelo grupo? Aqui é interessante colocar um fluxograma com tudo que o projeto terá, explicando o máximo de itens.

\subsection{Gerenciamento}
Como é feito o gerenciamento e quais são as funções de cada um dos integrantes? O que cada função executa?

\subsection{Ferramentas}
Quais ferramentas serão utilizadas? Como cada uma delas se encaixa no projeto?

\subsection{Lista de componentes}

Uma lista em formato de tabela deve ser colocada aqui. Ela deve conter, no mínimo: componente, descrição, quantidade, valor unitário, valor total.
\\

\begin{tabular}{c|c|c|c|c}
\hline
Componente & Descrição & Quantidade & Valor unitário & Valor Total\\
\hline
Arduíno & MEGA & 01 & R\$ 70,00 & R\$ 70,00 \\
\hline
\end{tabular}

\subsection{Cronograma}

Cronograma seguido pelo grupo. Vamos preencher em um primeiro momento e depois segui-lo. No Redmine o gráfico de Gantt é gerado automaticamente. Ele pode ser inserido aqui.

\section{O projeto de Automação implementado}

Este item não deve ter este nome. Ele deve ter o nome do projeto de cada grupo. Por exemplo: "Casa Inteligente".

Aqui vocês vão escrever o que implementaram.

\subsection{O sistema embarcado}

Como o arduíno foi ligado e a sua programação para este projeto.
Qual interface de comunicação foi implementada e seu código.
Quais interfaces de acionamento foram fisicamente implementadas? Exemplos: circuito para acionamento de servomotores, ou de motor de passo, ou de motores DC.

\subsection{O site - Acesso remoto aos dados}

Apresentar o site, com as informações que são lidas/escritas a partir do site que vocês vão implementar.

\subsection{O aplicativo para dispositivos móveis}

Apresentar o aplicativo, tela a tela, função a função, que foi criado.

\subsection{Especificações técnicas para funcionamento}

Neste item vocês devem colocar as restrições do projeto, as condições nas quais o projeto funciona (tamanhos de peças, distâncias das interfaces de comunicação, tipo de material reconhecido pelos sensores, dentre outras).

\section{Conclusões}

Conclua sobre todo o desenvolvimento do projeto, desde sua concepção até sua implementação. Quais foram as dificuldades encontradas e as soluções propostas?

Qual a importância da atuação de um grupo, interdisciplinar, no desenvolvimento de um projeto para a formação de cada um dos integrantes do grupo?

\section{Exemplos como inserir uma figura}

Clicar em Project, na barra de menus da página (acima e à esquerda). Lá procurar pela imagem que você quer colocar a partir do computador e fazer o upload dela aqui.

%%%%%%%%%%%%%%%%%%%%%%%%%%%%%%%%%%%%%%%%%%%%%%%%%%%%%%%%%%%%%%%%%%%%%%%%%%%%%%%%%%%%%%%%%%%%%%%%%%%%%%%%%%%%%%%%%%%%%%%%%%%%%%%%%%%%%%%%%%%%%%%%%%%%%%%%%%%%%%%%%%%%%%%%%%%%%%%%%%%%%%%%%%%%%%%%%%%%%%%%%%%%%%%%%%%%%%%%%%%%%%%%%%%%%%%%%%%%%%%%%%%%%%%%%
\textbf{EXEMPLO DE FIGURA A SER INSERIDA}
\begin{figure}[H]
	\begin{center}
 		%\includegraphics[width = 0.5\textwidth]{Imagens/ImagemExemplo}
 		%\captionof{figure}{\label{fig:ImagemExemplo}Descrição da imagem} 
	\end{center} 
\end{figure}
%%%%%%%%%%%%%%%%%%%%%%%%%%%%%%%%%%%%%%%%%%%%%%%%%%%%%%%%%%%%%%%%%%%%%%%%%%%%%%%%%%%%%%%%%%%%%%%%%%%%%%%%%%%%%%%%%%%%%%%%%%%%%%%%%%%%%%%%%%%%%%%%%%%%%%%%%%%%%%%%%%%%%%%%%%%%%%%%%%%%%%%%%%%%%%%%%%%%%%%%%%%%%%%%%%%%%%%%%%%%%%%%%%%%%%%%%%%%%%%%%%%%%%%%%

\textbf{EXEMPLO DE REFERÊNCIA A SER INSERIDA}
%\cite{IEEEreferencias:Ref1}
%%%%%%%%%%%%%%%%%%%%%%%%%%%%%%%%%%%%%%%%%%%%%%%%%%%%%%%%%%%%%%%%%%%%%%%%%%%%%%%%%%%%%%%%%%%%%%%%%%%%%%%%%%%%%%%%%%%%%%%%%%%%%%%%%%%%%%%%%%%%%%%%%%%%%%%%%%%%%%%%%%%%%%%%%%%%%%%%%%%%%%%%%%%%%%%%%%%%%%%%%%%%%%%%%%%%%%%%%%%%%%%%%%%%%%%%%%%%%%%%%%%%%%%%%

%%%%%%% Bibliografia %%%%%%%%
\addcontentsline{toc}{section}{Referências}  
\bibliography{bib/IEEEreferencias.bib} 
%%%%%%% Bibliografía %%%%%%%%    

\appendix  
\clearpage % o \cleardoublepage
\addappheadtotoc 
\appendixpage 

\section{Anexos 1.}

\section{Anexos 2.}
\end{document}