\section{État de l’art sur la sécurité Wi-Fi}
\label{sec:etatdelart}

La surface d’attaque des réseaux Wi-Fi n’a cessé d’évoluer depuis 1997. Les vulnérabilités historiques et contemporaines exploitent indifféremment : les faiblesses cryptographiques intrinsèques, les défauts d’implémentation, les erreurs de configuration, et les mécanismes d’interaction entre utilisateurs et points d’accès. Cet état de l’art se concentre exclusivement sur les \textbf{attaques}, désormais que les protocoles ont été décrits dans la section précédente, et présente une analyse détaillée des vecteurs connus, de leur fonctionnement interne et de leur pertinence actuelle.

%%%%%%%%%%%%%%%%%%%%%%%%%%%%%%%%%%%%%%%%%%%%%%%%%%%%%%%%%%%%%%%%%%%%%%%%%%%%%%%
\subsection{Évolution historique des attaques contre les réseaux Wi-Fi}

L’évolution des attaques Wi-Fi suit logiquement celle des protocoles eux-mêmes. Chaque nouvelle protection a donné naissance à un nouveau type d’attaque : statistiques pour WEP, cryptanalytiques pour WPA, protocolaires pour WPA2 et side-channels pour WPA3.

\newpage